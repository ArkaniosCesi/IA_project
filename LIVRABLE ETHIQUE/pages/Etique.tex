\section{Réalisation du Livrable Ethique}
\subsection{Introduction}
L'éthique en IA (Intelligence Artificielle) concerne les principes et les valeurs qui guident la conception, le développement, l'utilisation et la réglementation de la technologie de l'IA. Cela implique de prendre en compte les impacts éthiques et sociaux potentiels de l'IA, ainsi que de respecter les droits fondamentaux des individus et de la société dans son ensemble. L'éthique en IA se concentre sur des questions telles que la responsabilité, la transparence, la confidentialité des données, la non-discrimination, la sécurité, la durabilité et le bien-être social et environnemental. 

La mise en place de projets d'Intelligence Artificielle nécessite donc une réflexion approfondie sur les enjeux éthiques liés à cette technologie. En effet, les applications de l'IA peuvent avoir un impact significatif sur la société et les individus, allant de la vie privée à la discrimination en passant par les risques pour la sécurité et la santé. Afin de garantir une utilisation responsable et éthique de l'IA, la Commission Européenne a publié en 2019 une liste de 7 exigences recommandées.  

Pour appliquer ces exigences dans un projet IA, il est important de suivre une méthodologie rigoureuse pour identifier les risques éthiques potentiels et mettre en place des mesures de prévention. Dans ce document, nous allons présenter notre appropriation de la méthodologie proposée pour avoir une démarche éthique d'un projet IA. Nous allons décrire les différentes étapes à suivre et les outils à utiliser pour assurer une mise en œuvre responsable et éthique de l'Intelligence Artificielle. 

Dans le cadre de notre projet d'intelligence artificielle, nous avons travaillé sur un projet de prédiction du départ des employés à l'aide du Machine Learning. Le but de ce projet est de fournir des informations précieuses à notre client sur les employés qui pourraient être sur le point de quitter leur emploi. Cela permettra à notre client de prendre des mesures proactives pour retenir ces employés et ainsi réduire le taux de rotation du personnel. 

\newpage
\subsection{Méthodologie de réflexion concernant l'éthique}
\subsubsection{Situation initiale}
HumanForYou est une entreprise de produits pharmaceutiques basée en Inde qui emploie environ 4000 personnes. Chaque année, l'entreprise subit un turn-over d'environ 15% de ses employés, nécessitant de retrouver des profils similaires sur le marché de l'emploi. 

Afin de trouver des pistes d'amélioration pour retenir ses employés et réduire son taux de turn-over, la direction a fait appel à notre équipe de spécialistes de l'analyse de données. Nous devrons donc mener une analyse pour déterminer les facteurs ayant le plus d'influence sur le turn-over et proposer des modèles prédictifs pour améliorer la situation. Cependant, en tant qu'experts en analyse de données, nous devons également nous assurer que notre démarche est éthique et responsable, en prenant en compte les éventuels impacts sociaux et environnementaux de nos analyses et recommandations tout en respectant les sept exigences de l'Union Européenne. 

\subsubsection{Solutions}
Pour répondre à la problématique de la rétention des employés chez HumanForYou, nous avons identifié plusieurs solutions possibles. Tout d'abord, nous pourrions identifiez les facteurs qui conduisent les employés à quitter leur emploi : avec une première visualisation de données, nous avons constaté que sur l’année, un nombre conséquent de jeunes partent donc l'entreprise retient très mal les jeunes. Une solution pourrait consister proposer une augmentation de salaire aux employés âgés de moins de 30 ans afin de limiter leur départ, offrir aux employés des avantages supplémentaires ou des opportunités de carrière pour les encourager à rester. 
\newpage
\subsubsection{Partie prenante}
Dans le cadre de l'analyse des parties prenantes, il est important d'identifier les différentes personnes et entités impliquées dans la problématique du turn-over chez HumanForYou. Les parties prenantes clés incluent : 

\begin{enumerate}
    \item Les employés actuels et anciens de l'entreprise : ils ont une connaissance approfondie de l'entreprise et de son fonctionnement, et pourraient avoir des opinions sur les raisons du taux de turn-over élevé. 
    \item La direction de l'entreprise est également une partie prenante clé, car elle est responsable de la gestion globale de l'entreprise et peut influencer les politiques et pratiques qui affectent le turn-over. 
    \item Le service des ressources humaines est également important car il est responsable de l'embauche, de la formation et de la gestion des employés, et peut donc influencer le taux de turn-over. 
    \item Les employeurs potentiels sur le marché de l'emploi : ils seront influencés par la réputation de l'entreprise en matière de satisfaction des employés et pourraient être un indicateur des facteurs qui contribuent au taux de turn-over élevé. 
    \item Les clients de l'entreprise : ils pourraient être concernés par la qualité des produits fabriqués par l'entreprise et par la continuité des processus de production. 
    \item Les actionnaires de l'entreprise : ils s'intéressent aux résultats financiers de l'entreprise et pourraient être influencés par le taux de turn-over élevé qui peut affecter la rentabilité de l'entreprise. 
\end{enumerate}
En prenant en compte les besoins et préoccupations de ces parties prenantes, l'équipe d'analyse de données va pouvoir élaborer des modèles d'IA qui tiennent compte des facteurs clés qui contribuent au taux de turn-over élevé. 

\subsubsection{Impact}
L'étape des impacts permet d'identifier les conséquences potentielles des différentes solutions proposées pour les parties prenantes identifiées précédemment. 

Les impacts positifs des solutions sur les parties prenantes pourraient être une augmentation de la satisfaction et de l'engagement des employés, une réduction du taux de turn-over et une amélioration de la réputation de l'entreprise auprès des employés, des clients, des partenaires et des actionnaires. 

Cependant, il est important de noter que cette solution pourrait également entraîner des impacts négatifs tels qu'une augmentation des coûts pour l'entreprise ou une réduction de la productivité en raison d'une plus grande attention portée à la formation.  
\subsubsection{Filtre éthique}
Pour appliquer le filtre éthique à la problématique du turn-over au sein de l'entreprise pharmaceutique HumanForYou, nous avons pris en compte plusieurs critères éthiques importants. Tout d'abord, nous avons veillé à respecter la vie privée des employés en ne collectant que les données nécessaires à notre analyse. Nous avons également veillé à ce que les données ne soient pas utilisées à des fins discriminatoires, en excluant toute variable liée à l'âge, au sexe et à l'état civil.

Nous avons également pris en compte l'impact de nos résultats sur les employés eux-mêmes. Ainsi, nous avons évité de proposer des modèles qui pourraient mener à une pression accrue sur les employés pour qu'ils restent dans l'entreprise, en garantissant que nos suggestions se concentrent sur l'amélioration de l'environnement de travail. 

Enfin, nous avons pris en compte l'impact de notre analyse sur les parties prenantes externes de l'entreprise. Nous avons veillé à ce que nos propositions n'affectent pas négativement la réputation de l'entreprise ou la qualité de ses produits. 

Dans l'ensemble, notre analyse a été menée avec le plus grand souci de l'éthique, en veillant à ce que toutes les parties prenantes soient prises en compte et que notre recommandation soit équilibrée et respectueuse des droits de tous.

\newpage
\subsection{Les 7 exigences recommandées par la Commission Européenne }
\subsubsection{Respect de l'autonomie humaine}
Le respect de l'autonomie humaine implique que les systèmes d'IA doivent être conçus de manière à permettre aux employés de prendre des décisions éclairées et de garder le contrôle sur les données et les décisions prises par les systèmes d'IA. 

Pour l'application de l'exigence de respect de l'autonomie humaine pour des spécialistes de l'analyse de données dans le contexte de détermination des facteurs de turn-over dans une entreprise sont le respect des droits des employés à la vie privée et à la non-discrimination. En utilisant des méthodes d'analyse de données pour identifier les facteurs de turn-over, il est important de s'assurer que ces méthodes ne portent pas atteinte à la vie privée des employés ou ne conduisent pas à des décisions discriminatoires.  

Pour le projet, des données sensibles sur les employés sont utilisées pour l'analyse, il est important de mettre en place des mesures de sécurité pour protéger ces données et d'obtenir le consentement des employés avant leur utilisation. Au vu du fait que les résultats de l'analyse sont utilisés pour prendre des décisions concernant les employés, il est important de s'assurer que ces décisions sont justes et ne conduisent pas à une discrimination basée sur l'âge, le sexe, la race, la religion ou d'autres facteurs similaires. 

Pour répondre à cette exigence, un jeu de données dit éthique a été mis en place avec la suppression de certaines données en entrée discriminatoires telles que le sexe et l'âge des employés ou de données privées comme l'état civil. En respectant l'autonomie humaine, on garantit que les droits des employés sont protégés et que les décisions sont prises de manière équitable et non discriminatoire. 

\subsubsection{Robustesse technique et sécurité}
Les systèmes d'IA doivent être conçus de manière à assurer la fiabilité et la sécurité des résultats. Les algorithmes doivent être testés de manière rigoureuse pour éviter tout biais ou erreur de calcul. 

L'exigence de robustesse technique et de sécurité est pertinente dans le cas de la détermination des facteurs de turn-over dans une entreprise, vont être traiter des informations sensibles sur les employés de l'entreprise, telles que leur historique d'emploi, leur rémunération, leur formation, etc. Il est donc important de garantir la sécurité de ces données pour éviter toute violation de la vie privée des employés ou toute utilisation abusive de ces informations.  

De plus, la robustesse technique est essentielle pour garantir que les modèles d'analyse de données sont précis et fiables, et pour éviter les erreurs ou les biais qui pourraient entraîner des conséquences néfastes sur les employés et l'entreprise.  

Afin de mettre en pratique cette robustesse technique, nous avons utilisé divers classifier tel que le SGD, Random Forest, Decision Tree et le MLP. Après analyse de nos résultats, nous avons pu constater que l'algorithme Random Forest donne des résultats à 80 % de prédictions justes en moyenne.  

\newpage
\subsubsection{Confidentialité et gouvernance des données}
Les systèmes d'IA doivent être conçus de manière à garantir la confidentialité et la protection des données personnelles. Les employés doivent être informés sur la manière dont leurs données sont collectées et utilisées, et les règles de gouvernance des données doivent être claires. 

L'exigence de confidentialité et gouvernance des données peut être justifiée dans le cas d'utilisation du projet pour plusieurs raisons. Etant donnée que les données des employés de l'entreprise sont considérées comme sensible et privée. Il est donc important de garantir leur confidentialité et de mettre en place des mesures de sécurité pour empêcher leur accès non autorisé ou leur utilisation abusive. 

De plus, la mise en place de protocoles de gouvernance des données peut également aider à garantir l'exactitude et l'intégrité des données utilisées dans l'analyse, en s'assurant qu'elles sont collectées et traitées de manière responsable et fiable. 

Afin d'assurer cette confidentialité et gouvernance des données, nous avons élaboré une politique de confidentialité claire pour expliquer aux parties prenantes comment les données seront collectées, stockées et utilisées. Ainsi qu'évaluer régulièrement la politique de confidentialité et de gouvernance des données et la mettre à jour en fonction de l'évolution des risques et des normes éthiques. 

\subsubsection{Transparence}

Les systèmes d'IA doivent être conçus de manière à ce que les résultats soient compréhensibles et explicables. Les employés doivent pouvoir comprendre comment les décisions ont été prises, sur quelles données les algorithmes ont été formés, et comment les résultats ont été obtenus.  

Dans notre cas, la transparence est justifiée de plusieurs manières. Tout d'abord, les modèles d'IA utilisés pour identifier les facteurs doivent être transparents et compréhensibles pour les personnes impliquées dans le processus de décision. Cela garantit que les décisions prises sur la base des résultats de l'analyse de données sont justes et impartiales. 

La transparence peut également être justifiée pour permettre aux parties prenantes de comprendre comment les décisions sont prises et comment les résultats ont été obtenus. Cela peut renforcer la confiance des parties prenantes dans le processus et les résultats, ce qui peut être crucial pour le succès du projet. 

Afin de respecter la transparence, nous avons utilisé des techniques qui permettent de rendre les décisions du système explicables, telles que les modèles interprétables, les graphes de décision et les techniques d'exploration de données. Il est important de comprendre comment le modèle prend des décisions pour pouvoir expliquer son fonctionnement aux parties prenantes. On a donc élaboré un notebook détaillé accessible à tous les parties prenantes de l'entreprise afin que ceux-ci puissent avoir accès aux différents graffes, diagrammes et technique d'exploration des données du projet 


\subsubsection{Diversité, non-discrimination et équité}
Les systèmes d'IA ne doivent pas reproduire ou amplifier les biais ou les discriminations existants dans la société. Ils doivent être conçus de manière à promouvoir la diversité et l'égalité des chances pour tous, en évitant toute forme de discrimination basée sur le sexe, la race, l'orientation sexuelle, l'âge ou toute autre caractéristique personnelle. 

L'exigence de diversité, non-discrimination et équité est importante pour garantir que l'analyse des données ne soit pas biaisée envers certains groupes de personnes. Dans le cas du taux de turn-over d'une entreprise, il est important de s'assurer que les données utilisées pour l'analyse sont représentatives de l'ensemble des employés de l'entreprise, sans discrimination fondée sur des critères tels que l'âge, le genre, l'orientation sexuelle, l'origine ethnique, la religion, etc.  

De plus, il est important de veiller à ce que les résultats de l'analyse ne conduisent pas à des décisions discriminatoires ou injustes, telles que la mise en place de politiques qui pourraient avoir un impact disproportionné sur certains groupes d'employés.  

Par conséquent, c'est pour cette raison que toutes les données faisant référence à l'âge, le genre, l'orientation sexuelle, l'origine ethnique, la religion, etc ont été misent de côtés et qu'aucune décision concernant ces critères pour améliorer le résultat des analyses n'a été entrepris.

\newpage        
\subsubsection{Bien-être environnemental et sociétal}
Les systèmes d'IA doivent être conçus de manière à minimiser leur impact sur l'environnement et à maximiser leur contribution positive à la société. Les entreprises doivent prendre en compte les impacts sociaux et environnementaux de leurs technologies, et s'engager à contribuer au bien-être de la société dans son ensemble. 

L'exigence de bien-être environnemental et sociétal est justifiée pour déterminer les facteurs ayant le plus d'influence sur le taux de turn-over de l'entreprise. En effet, l'utilisation de données liées aux pratiques de travail de l'entreprise pourrait révéler des problèmes de santé ou de sécurité au travail qui pourraient affecter le bien-être des employés et contribuer au taux de turn-over. C'est donc pour cette raison que nous avons décidé de ne pas interagir avec les données horaires ainsi que les voyages d'affaires. 

\subsubsection{Responsabilité}
La responsabilité implique que les développeurs et les utilisateurs de l'IA doivent être en mesure de déterminer qui est responsable en cas de préjudice causé par un système d'IA. Cette responsabilité peut être partagée entre les différentes parties impliquées dans le cycle de vie de l'IA, notamment les concepteurs, les développeurs, les fournisseurs, les utilisateurs et les régulateurs. 

L'exigence de responsabilité s'applique à de nombreux aspects de l'analyse de données, y compris la collecte et l'utilisation des données. Dans le cas de la détermination des facteurs ayant le plus d'influence sur le taux de turn-over, nous devons nous assurer que les données utilisées sont fiables et représentatives de la population étudiée. Il faut également veiller à être transparents sur les méthodes utilisées pour l'analyse des données et les résultats obtenus, afin que les décisions basées sur ces résultats soient prises en toute connaissance de cause.  

En outre, il faut prendre en compte les conséquences potentielles des conclusions sur les employés et l'entreprise dans son ensemble, et être prêts à répondre de leurs actions en cas de problèmes ou d'effets imprévus.  

En fin de compte, les solutions que nous avons mise en place ont permis de contribuer à améliorer la situation, en maintenant un aspect fiable de par la robustesse technique, plutôt que de la compromettre tout en ayant une vision éthique en respectant les sept exigences recommandées par la Commission Européenne.







\newpage
\subsection{Conclusion}
En conclusion, il est essentiel de prendre en compte les aspects éthiques lors de la conception et de l'implémentation d'un projet d'IA. La méthodologie proposée par la Commission Européenne offre un cadre structuré pour s'assurer que les exigences éthiques sont respectées tout au long du processus.  

En tant que spécialiste de l'analyse de données, il est important de se montrer responsable et de prendre en compte les enjeux éthiques liés à l'utilisation de l'IA. En respectant les différentes exigences éthiques telles que la transparence, le respect de l'autonomie humaine, la confidentialité et la gouvernance des données, la diversité, la non-discrimination et l'équité, le bien-être environnemental et sociétal, et la robustesse technique et sécurité, nous pouvons nous assurer que notre projet d'IA est éthique et responsable. 

En fin de compte, les solutions que nous avons mise en place ont permis de contribuer à améliorer la situation, plutôt que de la compromettre tout en ayant une vision éthique en respectant les sept exigences recommandées par la Commission Européenne. 

En somme, intégrer les exigences éthiques dans notre projet d'IA permettra de maximiser ses bénéfices tout en minimisant ses risques. 