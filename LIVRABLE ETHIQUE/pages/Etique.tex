\section{Réalisation du Livrable Ethique}

La Commission Européenne a publié en 2019 une liste de 7 exigences recommandées pour garantir une IA éthique. Voici les détails de chaque exigence :

\subsection{Respect de l'autonomie humaine}
Les systèmes d'IA doivent être conçus de manière à respecter la volonté et les choix des personnes, en leur donnant la possibilité de contrôler l'utilisation de leurs données et en évitant toute forme de manipulation.
\newpage
\subsection{Robustesse technique et sécurité}
Les systèmes d'IA doivent être conçus de manière à assurer la fiabilité et la sécurité des résultats. Les algorithmes doivent être testés de manière rigoureuse pour éviter tout biais ou erreur de calcul.
\newpage
\subsection{Confidentialité et gouvernance des données}
Les systèmes d'IA doivent être conçus de manière à garantir la confidentialité et la protection des données personnelles. Les utilisateurs doivent être informés sur la manière dont leurs données sont collectées et utilisées, et les règles de gouvernance des données doivent être claires et transparentes.
\newpage
\subsection{Transparence}
Les systèmes d'IA doivent être conçus de manière à ce que les résultats soient compréhensibles et explicables. Les utilisateurs doivent pouvoir comprendre comment les décisions ont été prises, sur quelles données les algorithmes ont été formés, et comment les résultats ont été obtenus. 
\newpage
\subsection{Diversité, non-discrimination et équité}
Les systèmes d'IA ne doivent pas reproduire ou amplifier les biais ou les discriminations existants dans la société. Ils doivent être conçus de manière à promouvoir la diversité et l'égalité des chances pour tous, en évitant toute forme de discrimination basée sur le sexe, la race, l'orientation sexuelle, l'âge ou toute autre caractéristique personnelle.
\newpage
\subsection{Bien-être environnemental et sociétal}
Les systèmes d'IA doivent être conçus de manière à minimiser leur impact sur l'environnement et à maximiser leur contribution positive à la société. Les entreprises doivent prendre en compte les impacts sociaux et environnementaux de leurs technologies, et s'engager à contribuer au bien-être de la société dans son ensemble. 
\newpage
\subsection{Responsabilité}
La responsabilité implique que les développeurs et les utilisateurs de l'IA doivent être en mesure de déterminer qui est responsable en cas de préjudice causé par un système d'IA. Cette responsabilité peut être partagée entre les différentes parties impliquées dans le cycle de vie de l'IA, notamment les concepteurs, les développeurs, les fournisseurs, les utilisateurs et les régulateurs.
