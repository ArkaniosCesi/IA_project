\section{IA for HumanForYou}
\subsection{Contexte}
L'entreprise de produits pharmaceutiques HumanForYou basée en Inde emploie environ 4000 personnes. Cependant, chaque année elle subit un turn-over d'environ 15\% de ses employés nécessitant de retrouver des profils similaires sur le marché de l'emploi.

La direction trouve que ce niveau de turn-over n'est pas bon pour l'entreprise car :

\begin{itemize}
    \item Les projets sur lesquels étaient les employés quittant la société prennent du retard ce qui nuit à la réputation de l'entreprise auprès de ses clients et partenaires.
    \item Un service de ressources humaines de taille conséquente doit être conservé car il faut avoir les moyens de trouver les nouvelles recrues.
    \item Du temps est perdu à l'arrivée des nouveaux employés car ils doivent très souvent être formés et ont besoin de temps pour devenir pleinement opérationnels dans leur nouvel environnement.
\end{itemize}

Le direction fait donc appel à vous, spécialistes de l'analyse de données, pour déterminer les facteurs ayant le plus d'influence sur ce taux de turn-over et lui proposer des modèles afin d'avoir des pistes d'amélioration pour donner à leurs employés l'envie de rester.

\textbf{Données fournies}

Un certain nombre de données concernant les employés vous a donc été transmis par le service des ressources humaines.

Il s'agit de fichiers textes au format CSV.

Les données ont été anonymisées : un employé de l'entreprise sera représenté par le même EmployeeID dans l'ensemble des fichiers qui suivent.
\subsection{Attentes du livrable éthique}
Il s'agit dans un document mettant en évidence votre appropriation de la méthodologie proposée pour avoir une démarche éthique traçant vos choix justifiés par rapport aussi aux échanges qu'il y a eu dans votre équipe pendant toutes les phases du projet de la préparation du jeu de données à la proposition de pistes d'amélioration à votre client. En plus de la démarche, vous devrez bien mettre en évidence les décisions que vous avez prises dans l'équipe ainsi que les points de vigilance ou de contrôle à mettre en place par rapport aux problématiques éthiques rencontrées.

Pour vous guider dans cette démarche, vous pouvez vous appuyer les questions accessibles depuis ce site. Le fichier PDF téléchargeable donne les principales questions à se poser pour valider qu'un projet d'IA est éthique selon 7 exigences recommandées par la Commission Européenne :

\begin{itemize}
    \item Respect de l'autonomie humaine
    \item Robustesse technique et sécurité
    \item Confidentialité et gouvernance des données
    \item Transparence
    \item Diversité, non-discrimination et équité
    \item Bien-être environnemental et sociétal
    \item Responsabilité
\end{itemize}
